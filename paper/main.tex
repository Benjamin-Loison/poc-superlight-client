\documentclass[a4paper,11pt,oneside]{article}

% To use this template, you have to have a halfway complete LaTeX
% installation and you have to run pdflatex, followed by bibtex,
% following by one-two more pdflatex runs.
%
% Note thad usimg a spel chequer (e.g. ispell, aspell) is generolz
% a very guud ideo.

\usepackage[utf8]{inputenc}
\usepackage[a4paper,top=3cm,bottom=3cm,left=3cm,right=3cm]{geometry}
\renewcommand{\familydefault}{\sfdefault}
\usepackage{helvet}
\usepackage[english]{babel}     %% typographie française
\usepackage[style=numeric,language=english]{biblatex}
\usepackage{parskip}        %% blank lines between paragraphs, no indent
\usepackage[margin=1cm]{caption}%% give long captions a margin
\usepackage{booktabs}           %% typesetting nice tables
\usepackage[pdftex]{graphicx}   %% include graphics, preferrably pdf
\usepackage[pdftex]{hyperref}   %% many PDF options can be set here
\pdfadjustspacing=1     %% force LaTeX-like character spacing

\newcommand{\mylastname}{Agrawal}
\newcommand{\myfirstname}{Shresth}
\newcommand{\mynumber}{30002701}
\newcommand{\myname}{\myfirstname{} \mylastname{}}
\newcommand{\mytitle}{Super Light Clients for Ethereum Proof of Stake}
\newcommand{\myresearchsupervisor}{Dionysis Zindros}
\newcommand{\myinternalsupervisor}{Soren Petrat}

\hypersetup{
  pdfauthor = {\myname},
  pdftitle = {\mytitle},
  pdfkeywords = {},
  colorlinks = {true},
  linkcolor = {blue}
}

\addbibresource{refs.bib}

\begin{document}
  \pagenumbering{roman}

  \thispagestyle{empty}

  \begin{flushright}
    \includegraphics[scale=0.8]{bsc-logo}
  \end{flushright}
  \vspace*{40mm}
  \begin{center}
    \huge
    \textbf{\mytitle}
  \end{center}
  \vspace*{4mm}
  \begin{center}
   \Large by
  \end{center}
  \vspace*{4mm}
  \begin{center}
    \LARGE
    \textbf{\myname}
  \end{center}
  \vspace*{20mm}
  \begin{center}
    \Large
    Bachelor Thesis in Computer Science
  \end{center}
  \vfill
  \begin{flushleft}
    \large
    Submission: \today \hfill \\
    Research Supervisor: \myresearchsupervisor \\
    Internal Supervisor: \myinternalsupervisor \\
    \rule{\textwidth}{1pt}
  \end{flushleft}
  \begin{center}
    Jacobs University Bremen $|$ Department of Computer Science and Electrical Engineering
  \end{center}

  \newpage
  \thispagestyle{empty}

  \begin{center}
    \Large \textbf{Statutory Declaration}
    \vspace*{8mm}
  \end{center}

  \begin{center}
    \begin{tabular}{|l|p{85mm}|}
      \hline
      Family Name, Given/First Name & \mylastname, \myfirstname \\
      Matriculation number & \mynumber \\
      Kind of thesis submitted & Bachelor Thesis \\
      \hline
    \end{tabular}
    \vspace*{8mm}
  \end{center}

  \subsection*{English: Declaration of Authorship}
 
  I hereby declare that the thesis submitted was created and written
  solely by myself without any external support. Any sources, direct
  or indirect, are marked as such. I am aware of the fact that the
  contents of the thesis in digital form may be revised with regard to
  usage of unauthorized aid as well as whether the whole or parts of
  it may be identified as plagiarism. I do agree my work to be entered
  into a database for it to be compared with existing sources, where
  it will remain in order to enable further comparisons with future
  theses. This does not grant any rights of reproduction and usage,
  however.

  This document was neither presented to any other examination board
  nor has it been published.

  \subsection*{German: Erklärung der Autorenschaft (Urheberschaft)}
 
  Ich erkläre hiermit, dass die vorliegende Arbeit ohne fremde Hilfe
  ausschließlich von mir erstellt und geschrieben worden ist. Jedwede
  verwendeten Quellen, direkter oder indirekter Art, sind als solche
  kenntlich gemacht worden. Mir ist die Tatsache bewusst, dass der
  Inhalt der Thesis in digitaler Form geprüft werden kann im Hinblick
  darauf, ob es sich ganz oder in Teilen um ein Plagiat handelt. Ich
  bin damit einverstanden, dass meine Arbeit in einer Datenbank
  eingegeben werden kann, um mit bereits bestehenden Quellen
  verglichen zu werden und dort auch verbleibt, um mit zukünftigen
  Arbeiten verglichen werden zu können. Dies berechtigt jedoch nicht
  zur Verwendung oder Vervielfältigung.

  Diese Arbeit wurde noch keiner anderen Prüfungsbehörde vorgelegt
  noch wurde sie bisher veröffentlicht.

  \vspace{20mm}

  \dotfill\\
  Date, Signature

  \newpage

  \section*{Abstract}
  Blockchains are decentralised, trust-less and cryptographically secure ledger.  Ethereum is one of the most adopted Blockchains which allows for smart contract execution. The most common way to interact with Ethereum is through wallet applications. These wallet applications run on consumer hardware like PC, Browser or Smartphone. Such devices are constrained in storage and network bandwidth, hence the wallets don't actively participate in the consensus of the blockchain.  Instead they rely on external full-node service provider to get the state of the blockchain. This is counter intuitive to the principal of decentralization as these full-node service providers can behave maliciously or get hacked. There already exists constructions for verifying Proof of Stake is sub-linear complexity. These constructions use interactive bisection games between the nodes to quickly sync with the latest state of the network. In this paper we propose a similar constructions specifically tailored for ETH2 using Sync Committee.

  \newpage
  \tableofcontents

  \clearpage
  \pagenumbering{arabic}

  \section{Introduction}
%   This, like the rest, addresses fellow experts from your field (but
%   not from your particular topic of research). Here you should
%   technically connect to the main concepts from that field and give an
%   outline of your project, stating the research/engineering question
%   that you want to get answered by your project.

%   (target size: 1-2 pages)
 
  

  \section{Statement and Motivation of Research}

  This part should make clear which question, exactly, you are
  pursuing, and why your project is relevant/interesting. This is the
  place to explain the background and to review the existing
  literature. Where does your project extend the state of the art?
  What weaknesses in known approaches do you hope to overcome? If you
  have carried out preliminary experiments, describe them here.

  (target size: 5-10 pages)
  
  \subsection{Proof of Stake}
  Overview of components of proof of stake systems.
    
  \subsection{ETH2}
  ETH2 is the upcomming version of ETH1 which uses PoS. ETH2 merges two consensus protocols Casper FFG and LMD Ghost.
  \subsubsection{Time}
  Blocks, Epochs, etc
  \subsubsection{Casper FFG}
  Used for Finality. \cite{CasperFFG}
    
  \subsubsection{LMD GHOST}
  Secure only under good network conditions. \cite{Gasper}
    
  \subsection{PoPoS}
  Smartly uses bisection games to fast sync \cite{PoPos}
    
  \subsubsection{Bisection Game}
  Interactive games using Merkle Trees

  \section{Description of the Investigation}

  This is the technical core of the thesis. Here you lay out your how
  you answered your research question, you specify your design of
  experiments or simulations, point out difficulties that you
  encountered, etc.

  (target size: 5-10 pages)

  \section{Evaluation of the Investigation}

  This section discusses criteria that are used to evaluate the
  research results. Make sure your results can be used to published
  research results, i.e., to the already known state-of-the-art.

  (target size: 5-10 pages)

  \begin{table}[ht]
    \begin{center}
      \begin{tabular}{cl}
        \toprule
        Number & Description \\
        \midrule
        7 & A lucky number in Western culture \\
        8 & A lucky number in Chinese and other Asian cultures \\
        42 & Answer to the ultimate question of life, the universe, and everything \\
        404 & Not found \\
        \bottomrule
      \end{tabular}
      \caption{Useless insights I gained with no further meaning}
    \end{center}
  \end{table}
  
  \begin{figure}[ht]
    \begin{center}
      \includegraphics[width=.8\textwidth]{bsc-plot}
    \end{center}
    \caption{Many dots distributed over a two dimensional unit space
      without any discernible pattern or deeper meaning}
  \end{figure}

  \section{Conclusions}

  Summarize the main aspects and results of the research
  project. Provide an answer to the research questions stated earlier.

  (target size: 1/2 page)

  \nocite{JS06}

  \newpage
%   \bibliographystyle{unsrt}
%   \bibliography{bsc-sample}
  \printbibliography

\end{document}